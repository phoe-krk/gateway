\documentclass[10pt]{extarticle}
\usepackage{extsizes}
\usepackage[utf8]{inputenc}
\usepackage[polish]{babel}
\usepackage[top=1in, bottom=1in, left=1in, right=1in]{geometry}
\usepackage{lipsum}
\usepackage{indentfirst}
\usepackage{titlesec}
\usepackage{enumitem}
\usepackage{setspace}

\titleformat{\subsection}
  {\normalfont\Large\bfseries}{\thesubsection}{1em}{}[{\titlerule[1.2pt]}]
\titleformat{\subsubsection}[block]
    {\bfseries}{\hspace{0.4cm}\thesubsubsection}{1ex}{}[{\titlerule[0.4pt]}]

\usepackage{multicol}
\setlength{\columnseprule}{0.4pt}

\font\tt=rm-lmtl10
\font\itt=rm-lmtlo10
\font\btt=rm-lmtk10
\font\bitt=rm-lmtko10

\newcommand{\genericentry}[4] {\setstretch{0.7}
                               \vspace{-0.5em} \subsubsection{{\btt #1} \hfill {\it #2} \\
                               {\vspace{0.0em} \hspace{0.2cm} \scriptsize \ttfamily #3 $\rightarrow$ #4}}
                               \setstretch{1.0}}
\newcommand{\classentry}[2] {\subsubsection{{\btt #1} \hfill {\it #2}}}

\newcommand{\accessorentry}[4] {\setstretch{0.7}
                               \vspace{-0.5em} \subsubsection{{\btt #1, (SETF #1)} \hfill {\it #2} \\
                               {\vspace{0.0em} \hspace{0.2cm} \scriptsize \ttfamily #3 $\rightarrow$ #4}}
                               \setstretch{1.0}}

\begin{document}
%\begin{multicols*}{2}
%[
\section{Gateway server internal protocol}
This describes the internal protocol for the Gateway server, which is written in Common Lisp.
%]



\subsection{Sexpable}
An object is sexpable if:
\begin{itemize}[noitemsep]
\item it provides a method for the {\btt SEXP} generic function;
\item it should ever be output from/to the server so that its representation is reconstructible in the client.
\end{itemize}
\genericentry{SEXP}{Generic Function}{object}{sexp}
This function takes a sexpable object and produces its representation as an S-expression. \par
The resulting S-expression may contain only lists, symbols, numbers and strings. \par
This representation must not refer to any system-specific details of the object, such as its temporary memory location on the machine and/or any IDs internal to the runtime and not the persistent data store.




\subsection{Password}
A class fulfills the password protocol if:
\begin{itemize}[noitemsep]
\item it provides a method for the {\btt MAKE-PASSWORD} and {\btt PASSWORD-MATCHES-P} generic functions;
\item it is the {\btt PASSWORD} class itself or a subclass of it.
\end{itemize}
\classentry{PASSWORD}{Class}
The PASSWORD class provides a default implementation of the password protocol. It may be overridden by a subclass's own behaviour that suits the subclass's own implementation.
\genericentry{MAKE-PASSWORD}{Generic Function}{passphrase}{password}
This creates a newly allocated instance of the {\btt PASSWORD} class, for which {\btt(password-matches-p instance passphrase)} returns {\btt T}.
\genericentry{PASSWORD-MATCHES-P}{Generic Function}{password passphrase}{generalized-boolean}
Returns {\btt T} if {\btt passphrase} matches the {\btt password}; otherwise, returns {\btt NIL}.




\subsection{Chatter}
A class fulfills the chatter protocol if:
\begin{itemize}[noitemsep]
\item it provides a method for the {\btt SEND-MESSAGE} and {\btt MSG} generic functions;
\item it is a subclass of the {\btt CHATTER} class.
\end{itemize}
\classentry{CHATTER}{Protocol Class}
\genericentry{SEND-MESSAGE}{Generic Function}{message recipient}{nil}
This sends the provided message, which is a generalized instance of the {\btt MESSAGE} class, to the provided recipient, which is a generalized instance of the {\btt CHATTER} class. \par
For {\btt MSG}, see the {\bf Message} protocol.





\subsection{Message}
A class fulfills the message protocol if:
\begin{itemize}[noitemsep]
\item it provides a method for the {\btt SENDER}, {\btt RECIPIENT}, {\btt DATE-OF}, {\btt CONTENTS} and {\btt MSG} generic function;
\item it is a subclass of the {\btt CHATTER} class.
\end{itemize}
\classentry{MESSAGE}{Class}
The MESSAGE class provides a default implementation of the message protocol. It may be overridden by a subclass's own behaviour that suits the subclass's own implementation.

\genericentry{SENDER}{Generic Function}{message}{chatter}
\classentry{SETF SENDER}{Generic Function}
Accesses the sender of the {\btt message}. The sender must be a generalized instance of the {\btt CHATTER} class.

\genericentry{RECIPIENT}{Generic Function}{message}{chatter}
\classentry{SETF RECIPIENT}{Generic Function}
Accesses the recipient of the {\btt message}. The recipient must be a generalized instance of the {\btt CHATTER} class.

\genericentry{DATE-OF}{Generic Function}{message}{chatter}
\classentry{SETF DATE-OF}{Generic Function}
Accesses the date of the {\btt message}. The date must be a generalized instance of the {\btt DATE} class.

\genericentry{CONTENTS}{Generic Function}{message}{chatter}
\classentry{SETF CONTENTS}{Generic Function}
Accesses the contents of the {\btt message}. The contents must be a {\btt STRING}.

\genericentry{MSG}{Generic Function}{sender recipient contents}{message}
This creates a newly allocated instance of the {\btt MESSAGE} class, using the provided {\btt sender} (a {\btt CHATTER}), {\btt recipient} (a {\btt CHATTER}), {\btt contents} (a {\btt STRING}) and the current system date. \par
For {\btt SEND-MESSAGE}, see the {\bf Chatter} protocol.





\subsection{Persona}
A class fulfills the persona protocol if:
\begin{itemize}[noitemsep]
\item it provides a method for the {\btt NAME}, {\btt PLAYER} and {\btt FIND-PERSONA} generic function;
\item it is a subclass of the {\btt PERSONA} class.
\end{itemize}
\classentry{PERSONA}{Class}
The PERSONA class provides a default implementation of the persona protocol. It may be overridden by a subclass's own behaviour that suits the subclass's own implementation. \par
Each persona can be uniquely described by its {\btt name}. It is an error to try to instantiate a persona whose name is {\btt STRING=} to another persona already existing in the database.

\accessorentry{NAME}{Generic Function}{persona}{name}
Accesses the name of the {\btt persona}. The contents must be a {\btt STRING}.

\accessorentry{PLAYER}{Generic Function}{persona}{name}
Accesses the player of the {\btt persona}. The contents must be a {\btt PLAYER}.

\genericentry{FIND-PERSONA}{Generic Function}{name}{persona}
Returns the {\btt persona} whose name is {\btt STRING=} to {\btt name}.

%\end{multicols*}
 
\end{document}